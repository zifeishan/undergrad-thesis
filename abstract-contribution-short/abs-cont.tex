\section{摘要}

体育比赛中,如何对选手的能力进行排名,是一个富有价值的研究方向。 然而,这一课题尚未被学术界广泛研究。
各项体育比赛中,大多传统的排名方法只考虑了线性的数据,均未考虑选手之间的关系。 为解决这一问题,在近几年新兴的网络理论的指导之下,
我们提出了竞技体育的通用的抽象模型。 在我们的比赛模型中,选手担任一种或多种角色, 不同角色的选手之间,存在不同种类的竞争关系。
我们以选手作为结点、选手之间的竞争作为边,建立比赛的网络模型。 我们设计了比赛网络中通用的算法
\emph{GameRank},来对选手的某种竞争能力进行排名。 我们的算法采用了随机行走模型,以考虑选手之间关系对选手排名的影响。
算法的指导思想为:强大的选手能够战胜强大的对手。 我们应用这一方法,对实际的棒球比赛进行了排名,评估选手的打击、投球能力。
为了评估算法,我们将GameRank的结果与其他现有排名方法进行了比较,说明我们的方法不仅能得到与现有权威排名相似的结果,
而且我们的排名结果更加有序,更满足``高位选手更难战胜''的规律。

这一工作的贡献有三方面:一、创新地提出了通用竞技比赛的网络模型;二、提出了更准确的选手排名的算法;三、进行了大规模棒球数据的网络分析和排名,发现了诸多规律和现象。

\section{综述和主要贡献}

体育比赛是许多人娱乐生活中不可或缺的部分。 体育比赛的分析,有助于评估球员和球队,预测比赛结果和球员水平,对制定球队策略也有重要影响。
在各种分析中,球员水平的排序可以提供直观、定量的参考,对观众了解球员水平、球队管理球员等都有深远意义。
目前,不同类型的体育比赛,都有各自的球员排名方法。
然而,他们的排名通常基于独立的线性数据,缺乏一种考虑球员之间联系的全局观点,不能反应竞争对手的实力对球员评估的影响。
例如,棒球中``安打率''这一数据仅考虑打击员所有打席中有几个安打,却不考虑打击员面对投手的能力,而实际中,面对越强的投手能打出安打的球员,越是能力出众的球员。
此外,不同类型的竞技比赛之间有许多共通点,但没有一个有力的、统一的球员排名模型,能够利用这些共通点。

在本文中,我们展示了一个通用的竞技体育网络模型,其中结点为选手,边为选手之间的竞争关系,不同角色的选手之间存在不同的竞争。
在这个网络上,我们设计了GameRank---一个通用的基于竞争关系的选手排序算法, 它的基本思想为:强大的选手能够战胜强大的对手。
GameRank应用了一个随机行走模型,来考虑对手的实力对评估选手的影响,以准确把握这一基本思想。

我们将 GameRank 应用在棒球比赛上,分析了美国职业棒球大联盟(MLB)的大规模数据集。
我们解析数据集生成了比赛网络,评估了球员的投球和打击能力,将1921年到2012年的选手投球和打击能力加以排名。

我们还应用这一棒球数据集对GameRank的效果进行了评估。评估方法为,比较GameRank和其他现行的著名排名方法的排名结果。结果从多个方面表明GameRank有出色的表现:
一、 GameRank与其他排名方法产生了相似的结果。 二、
通过分析不同投手和打击员交手的胜率,进行逆序对计数,我们发现GameRank更满足``高排名的对手比低排名的对手更难战胜''这一规律,因此它比其他排名方法更加有序。
三、 GameRank使用了强大的网络模型,可以考虑球员之间的关系。 四、
GameRank可以将所有球员进行排名,而有些排名方法只能排名部分球员。

接着,我们使用排名信息分析了棒球网络的数据。我们发现:

\begin{enumerate}
\item
  球员水平在逐年接近。
\item
  分析投手的打击能力:投球能力越高的投手,打击能力往往也越高。
\item
  棒球网络中的结点数(球员数)和边数都在逐年增长,但网络密度在逐年减小。
\item
  大部分进入GameRank前十名球员,只在前十榜中只出现过一次,但历史上有10名球员上榜了超过10次。
\item
  1968---1983 年间出道的球员中,上过前十榜的球员异常地少。
\end{enumerate}
\quad 我们的贡献主要体现在以下方面:

\begin{enumerate}
\item
  我们创新地提出了通用竞技比赛的网络模型;
\item
  我们提出了更准确的选手排名的算法,创新地考虑了对手实力对选手评估的影响;
\item
  我们进行了大规模棒球数据的网络分析和排名,发现了诸多规律和现象。
\end{enumerate}
